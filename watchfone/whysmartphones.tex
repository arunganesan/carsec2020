\watchfone{} uses smartphone sensors for cross-validation of vehicle sensors. Smartphones are attractive to play this role for several reasons.
\begin{itemize}
    \item They are powerful and capable computers with multiple ways to sense the outside environment. For example, the latest flagship phone from Samsung has a 2.7/1.8 Ghz 8-core processor and 11 sensors \cite{lee_samsung_2017}.
    \item They are ubiquitous and always-on-and-with-person. One study found that 77\% of U.S.~adults have smartphones \cite{pew_research_center_demographics_2018}. Given their prevalence, smartphones can act as a free source of 
    data redundancy in our vehicular IDS.
    \item Smartphones undergo a frequent refresh rate and 
    on average, users buy a new smartphone every $\approx$2 years \cite{kollewe_dixons_2017}. 
   In a span of two years, phones often undergo significant improvements in their resources including CPU, memory, OS, sensors, etc.
   For example, the Galaxy S series significantly improved in processing power (2.3 Ghz quad core in 2016 to 2.7/1.8 Ghz octa core in 2018) and software (Android 6 in 2016 to Android 8 in 2018). 
\end{itemize}