% !TeX root = ./main.tex

Discussed below are a couple of remaining issues worth 
further investigation beyond our current approach.

\begin{itemize}
   
    \item \textbf{Compromised Phone Sensors.} As described in our adversary 
    model (Sec.~\ref{background/advmodel}), we assume that the phone isn't 
    compromised. Mobile security is an active research area on its own and is 
    orthogonal to our work. We refer the reader to recent surveys on mobile 
    security \cite{faruki_android_2015}.
    
    \item \textbf{Additional Redundancy.} We demonstrated that \watchfone{} can run 
    on passengers' phones while they engage in common activities on their 
    phones. Similarly, we can extend this work to other devices which have the 
    required three sensors. \watchfone{} only needs IMU, magnetometer, and GPS 
    sensors to estimate vehicular sensors. Many existing IoT devices could be 
    used in this way including smartwatches, fitness trackers, or some mounted 
    after-market devices such as Amazon Echo Auto,
    \footnote{\url{https://www.amazon.com/Introducing-Echo-Auto-first-your/dp/B0753K4CWG}} 
    as long as they expose access to these three sensors. We leave the evaluation 
    of this extension as future work.
     
    \item \textbf{Response After Detection.} \watchfone{} serves as a two-factor 
    source of knowledge to detect sensor falsification attacks. If an attack is 
    detected, \watchfone{} merely notifies the driver of the suspicious activity and
    they can choose to take the car to a mechanic for further diagnostics. We 
    do not consider automated response based on this information. Automatically 
    responding is risky since the attacker may target the smartphone in order
    to trigger this response mechanism and immobilize the car. Therefore, we
    restrict \watchfone{} to \textbf{\textit{detection}} and leave
    \textbf{\textit{response}} up to the driver.
\end{itemize}