% !TeX root = ./main.tex

Vehicles are built with many ECUs which are responsible for different vehicular
functions and usually communicate with each other using one or more Controller 
Area Network (CAN) buses. 
These buses are broadcast-based and use a decentralized protocol for access
 arbitration and real-time communication. %In some cases, ECUs are directly connected to each other --- such is the case for the collision-prevention system and the anti-lock braking 
%system in some vehicles \cite{miller_advanced_2016}.

Sensor-falsification attacks use these communication networks to broadcast 
sensor data masquerading as one of the ECUs. Since the CAN protocol does not
identify/authenticate senders/receivers, an attacker capable of writing to CAN
bus can achieve this 
\cite{koscher_experimental_2010,miller_adventures_2013,miller_advanced_2016}. 
For example, an attacker can gain write access using an 
Internet-facing Infotainment system, forced Bluetooth pairing or even a 
corrupted audio file with a firmware update \cite{checkoway_comprehensive_2011}. 
Once successful, the attacker can mount an attack which involves writing 
falsified messages to the CAN bus. The intention of this attack may be to 
confuse the user (e.g., \cite{miller_adventures_2013} p.32, 
\cite{koscher_experimental_2010} p.456), or to control the vehicle 
(e.g., \cite{miller_advanced_2016} p.27), as enumerated in Table~\ref{tab:attacks} 
and categorized in Fig.~\ref{fig:taxonomy}.

\bgroup
\def\arraystretch{\tablepadding}
\begin{table*}
\centering
\scriptsize
\begin{tabu} to 0.45\textwidth {|c|c|c|c|c|c|}
	\hline
    \rowcolor{black!15}
     \textbf{Ref.} & \textbf{ID} & \textbf{Page \#} & \textbf{Target ECU} & \textbf{Sensor spoofed} & \textbf{Purpose of attack} \\ \hline
     & 1 & 32 & Instrument Cluster (IC) & Speed & Confuse user \\ \cline{2-6}
     & 2 & 32 & IC & Engine RPM & Confuse user \\ \cline{2-6}
     & 3 & 34 & IC & Odometer & Confuse user \\ \cline{2-6}
     & 4 & 38 & Pre-Collision System (PCS) & RADAR-state & Sudden brake, disable gas pedal \\ \cline{2-6}
     & 5 & 43 & Intel. Park. Assistant System (IPAS) & Gear, Speed & Control steering \\ \cline{2-6}
     & 6 & 45 & Lane Keeping Assistant (LKA) & Camera-state & Control steering \\ \cline{2-6}
     \multirow{-7}{*}{\cite{miller_adventures_2013}} & 7 & 59 & IC & Fuel gauge & Confuse user \\ \hline
     %
     %
     & 8 & 455 & IC & Fuel gauge & Confuse user \\ \cline{2-6}
     \multirow{-2}{*}{\cite{koscher_experimental_2010}} & 9 & 456 & IC & Speed  & Confuse user \\ \hline 
     %
     %
     & 10 & 84 & IC & Engine RPM & Confuse user \\ \cline{2-6}
     \multirow{-2}{*}{\cite{miller_remote_2015}} & 11 & 85 & Anti-lock Brake System (ABS) & Imminent collision state  & Cause ABS to apply brakes \\ \hline
     %
     %j
	& 12 & 7 & IC & Speed & Confuse user \\ \cline{2-6}
    & 13 & 23 & Power Steer. Ctrl. Module (PSCM) & Engine RPM & Put ECU in diagnostic mode \\ \cline{2-6}
     \multirow{-3}{*}{\cite{miller_advanced_2016}} & 14 & 27 & Park. Assistance Module (PAM) & Speed & Control steering \\ \hline
     \end{tabu}
    \caption{Example CAN bus injection attacks which require falsifying vehicular sensors.}
    \vspace{-0.3in}
    \label{tab:attacks}
\end{table*}
\egroup