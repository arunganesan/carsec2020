% !TeX root = ./main.tex
%Background + A pressing need: Car security is in danger
Cyber attacks on cars are now a very real and serious threat. 
In recent years, researchers have demonstrated numerous 
vulnerabilities which allow an attacker to take control of 
a vehicle and put drivers and passengers at peril \cite{checkoway_comprehensive_2011,foster_fast_2015,miller_adventures_2013,miller_advanced_2016}. 
Most of these attacks involve injecting \textit{falsified sensor data} 
into the Controller Area Network (CAN), the \textit{de facto} 
communication network within a vehicle, and hence the first and foremost
step in their defense is to detect these falsified sensor values.

% The pressing need
While there are many proposals of increased security in future vehicles, 
we need to enhance security and safety in cars on the roads today. In 
particular, we need a system which can be used immediately by passengers
to enhance their sense of confidence 
in the security and safety of their vehicle.

% Why others failed to fill the need: They fail to bring it to cars NOW
\textbf{State-of-the-art.} Existing security assistant proposals fail to
meet this pressing need of increased security in today's cars. They usually
detect these in-vehicle attacks by modeling packet-arrival characteristics 
within the vehicle \cite{cho_fingerprinting_2016,muter_structured_2010} or 
the relationships between in-vehicle sensors \cite{cho_cps_2015,ganesan_exploiting_2017}. 
If the vehicle is compromised, both classes of security assistants suffer 
from untrustworthy data since they both rely on in-vehicle sensor data. 
There has also been a strong commercial push for vehicular intrusion detection 
systems, but they all require deep integration with the vehicle hardware, 
making their integration with existing vehicles prohibitively expensive \cite{arilou_automotive_2018,argus_cyber_security_argus_2018,towersec_harman_2018}. 

\begin{figure}[t]
  \includegraphics[width=\textwidth]{watchfone/figures/frontfig-2.pdf}
  \caption{Three smartphone sensors are used to infer six 
  different vehicular properties. \watchfone{} compares these 
  inferred values with those reported by the vehicle to detect \textit{sensor-falsification attacks}.}
  \label{fig:estimate-sensors-pic}
\end{figure}

% How you fill the need: Carsec - Phone as a Car Security Assistant
\textbf{Proposed Solution.} We propose a new security assistant called \watchfone\ to fill this need. 
The first step of many of the vehicular attacks reported in literature involves injecting \textit{falsified sensor data} 
into the CAN bus. \watchfone{} independently acquires these sensor values using 
the \textit{smartphone's sensors} and cross-validates them with what is seen in 
the CAN bus. By doing so, \watchfone{} adds a second layer of assurance in the
vehicle's sensor values on CAN, and can quickly identify when \textit{sensor 
falsification} is taking place. \watchfone{} can be used in tandem with other 
security measures to detect and defend against the emerging threat of vehicular 
cyber attacks. 

Smartphones have proven successful in assisting with other vehicular tasks such as dangerous driving detection 
\cite{hong_smartphone-based_2014, meseguer_drivingstyles:_2013, li_dangerous_2016, yu_fine-grained_2017, enriquez_canopnr:_2012}, road monitoring \cite{eriksson_pothole_2008, hu_smartroad:_2015, seraj_roads:_2016, zheng_u-air:_2013, ganti_greengps:_2010}, and trajectory mining \cite{coric_crowdsensing_2013, wang_crowdatlas:_2013, carisi_enhancing_2011, liu_mining_2012}. 
We expand the use of smartphones to enhance \textit{vehicular security} by detecting sensor-falsification attacks. 
To the best of our knowledge, this is the first application of smartphones for vehicular security assistance.

One of the main advantages of \watchfone{} is its ease of deployment 
at no additional cost. A driver simply installs the \watchfone{} app from the app 
market, and pairs it with a Bluetooth OBD dongle which reads values from the 
vehicle's CAN bus.  The {\em read-only} OBD dongle simply relays the information 
it sees in the CAN bus to the phone, thereby not contributing to the attack surface. 
% compromised dongles are addressed in the adversary model section
\watchfone{} runs automatically in the background to independently acquire vehicle 
sensors on the CAN bus via a dongle and cross-validate them. Since almost everyone 
carries a smartphone these days, we can use both driver's and passenger's phones to
increase the available sensor redundancy at no additional cost. 

% One of the main selling points of this paper
By implementing \watchfone{} on a smartphone, we provide this additional security 
to drivers today. The smartphone serves as an external source of knowledge about 
vehicular dynamics, thereby overcoming the above-mentioned limitation of sole-reliance
on in-vehicle sensor data. Furthermore, \watchfone{} capitalizes on the existing 
hardware and computing power of smartphones. This can be used in addition to commercial
solutions, if any, which provide enhanced protection but require deeper and more
expensive integration with the vehicle. 

% Why is your solution interesting and challenging
\textbf{Key Challenges and Solutions.} In order to use smartphones to detect vehicular
sensor-falsification attacks, we have to estimate vehicular values and compare them 
with the sensor values on the CAN bus.


\paragraph{Challenges in Evaluating Falsification Attacks} There exists a dearth of 
evaluation metrics for CAN-bus injection or falsification attacks, even though injection 
attacks are often a necessary step before achieving vehicular compromise. To solve this 
lack of benchmarks, we survey existing literature on vehicular cyber-attacks and identify
some of the more common sensor-falsification attacks. We categorize this into different 
types of falsification attacks, and evaluate \watchfone{} against these attacks to demonstrate
its effectiveness in a realistic scenario.

\paragraph{Challenges in Estimating Vehicular Sensor Values} 
\changes{Estimating the readings of vehicle sensors from smartphone sensors requires a robust method which is resilient to common phone usage. Prior work on vehicle sensor estimation requires data from installed sensors and has not addressed the difficulties associated with common phone usage/movements \cite{liu_bigroad:_2017, skog_indirect_2014}. 
In contrast, using the vehicle's physical dynamics models (e.g., Ackerman model 
\cite{liu_bigroad:_2017}), publicly available vehicle specifications from 
the OEMs (e.g., gear ratios), and sensor fusion techniques (e.g., complementary 
filter, neural network), we have built a system that is robust to common phone usage/movements. 
We demonstrate this through extensive experiments.}

\changes{Furthermore, to overcome the lack of a well-defined vehicle model for estimating the current gear position, we have developed a neural network to predict the gear position given the change in recent speed.}


% Key results:
\textbf{Key Findings.} 
We have implemented \watchfone{} in Android and evaluated it against sensor-falsification
attacks. We evaluated \watchfone{} by injecting three types of sensor-falsification attacks reported in 
literature --- {\em sudden}, {\em gradual}, and {\em delta} injections --- in real-world driving traces.  
\watchfone{} is shown to be capable of detecting falsifications of 6 vehicular sensors 
with different levels of true positive rate (TPR), ranging from \suddenXspeedXTPRforFPRHalfPercent{}\% 
for speed sensors to \suddenXrpmXTPRforFPRFivePercent{}\% for RPM falsifications,
with very low false positive rate (FPR), 
often set to less than 1\% FPR. 
\watchfone{} can most accurately detect speed,  gear position, fuel and odometer falsifications.

Furthermore, \watchfone{} only consumes $\approx$8\% of the CPU on average, even while performing expensive neural 
network-based sensor estimation algorithms.

This paper makes the following main contributions:

\begin{itemize}

\item {\em Thorough characterization of CAN-bus sensor falsification attacks}. We survey the literature on vehicular cybersecurity to create a taxonomy of different attacks on vehicular CAN bus. This taxonomy of attacks will guide future research into defenses against such attacks.

\item {\em Extensive evaluation of vehicle sensor estimation algorithms under diverse realistic situations}. We show the robustness of vehicle sensor estimation algorithms on 6 different vehicles, 912 miles of driving, and under 4 different common uses of the phone by 7 different passengers. 


\item \changes{{\em Important extension of prior vehicle estimation research}. We extend existing vehicle 
estimation research by (a) developing a novel neural-network based gear-estimation algorithm and (b) investigating 
the failure modes of engine RPM estimation using prior methods.}

\item {\em Development and evaluation of \watchfone\, a novel system to detect CAN-bus injection attacks}. We use phone-based vehicle sensor estimation to detect CAN-bus injection attacks. To the best of our knowledge, this is the first to apply phone-based vehicle estimation algorithms for the detection of CAN-bus sensor falsification attacks. We also demonstrate its computational feasibility with an efficient Android implementation and provide source code facilitating future follow-up research in this area. 
\end{itemize}

\textbf{Paper Organization.} The paper is structured as follows. 
Sec.~\ref{related} reviews other related IDS solutions while Sec.~\ref{background} presents the necessary background and 
attack model. Next we describe \watchfone{} in Sec.~\ref{system} 
and evaluate it against simulated and realistic attacks in Sec.~\ref{eval}. Finally, we discuss limitations of current \watchfone{} and future work in Sec.~\ref{discussion},
and conclude the paper in Sec.~\ref{conclusion}. 