% !TeX root = main.tex

We model our adversary after what is normally found in related literature. At 
the bare minimum, the adversary has write access to the CAN bus (e.g., through 
the UConnect infotainment system, as demonstrated in \cite{miller_remote_2015})
and uses this access to broadcast falsified sensor values. S/he uses falsified 
sensor readings to confuse the user or force ECUs within the vehicle to misbehave. 
The attacks can vary in how quickly they are accomplished and must be detected. 

\subsubsection{Trusted Components}
In order to run \watchfone, the driver must carry a smartphone equipped with IMU and 
GPS sensors. We assume the smartphone is not compromised and is mounted on the 
windshield or placed in the cup holder.  Smartphone security is an active area 
of research and there exist numerous deployed solutions to ensure that the software 
running on a smartphone has integrity and hasn't been compromised. Besides, 
securing smartphones is orthogonal to our work.

Additionally, the driver needs to connect an OBD dongle to the OBD-II port found 
under the steering wheel. These are cheap and widely available devices used by many 
people for diagnostic and telematics purposes. We assume that the OBD-dongle is not 
compromised. We envision an OBD dongle which has very limited functionality and only 
serves to dutifully relay the information it reads on the CAN bus. A simple OBD dongle 
which only serves the purpose of relaying information doesn't need Internet connectivity 
and doesn't need programmability.  Although prior work has shown remote compromise 
through the OBD dongle \cite{foster_fast_2015, miller_remote_2015}, its target was 
dongles that have Internet connectivity and sophisticated programmability. This 
sophistication makes them vulnerable to attacks from the Internet. 

\subsubsection{Attacks to Be Covered}
\watchfone{} detects \textit{sensor falsification attacks} which are mounted via the CAN bus 
that can be externally cross-validated using the smartphone sensors. Sensor falsification 
on the CAN bus is often the first step in vehicular cyber-compromise reported in literature 
\cite{miller_advanced_2016, miller_adventures_2013, miller_remote_2015,koscher_experimental_2010}. 
As such, it is important to detect and thwart the attacks at this stage before they 
progress further.

We assume the sensor-spoofing happens through CAN-bus injection. In other words, it 
doesn't involve externally spoofing the sensor values via LiDAR injection or other means. 
There are numerous examples of CAN-bus injection attacks reported in literature. For 
example, \cite{koscher_experimental_2010} p.458 demonstrates a compound attack which 
involves falsifying a speedometer reading and \cite{miller_adventures_2013} p.43 presents 
a detailed attack that controls the steering wheel angle by falsifying gear and vehicle 
speed readings. 

\subsubsection{Attacks Not to Be Covered}
\watchfone{} is not designed to detect CAN-injection attacks which do not involve sensor
falsification. For instance, \watchfone{} cannot detect CAN-injections used to lock/unlock the 
door found in \cite{miller_adventures_2013} p.58. Additionally, \watchfone{} is not designed 
to detect CAN-injections where the spoofed sensor value reflects the actual vehicle 
sensor value. This happens when the spoofed value is the same as the real value or when 
the vehicle reacts to the spoofed value and changes its state to resemble the input value 
(e.g., \cite{miller_advanced_2016} p.22). 


\begin{figure}[h]
  \includegraphics[width=0.45\textwidth]{watchfone/figures/taxonomy.pdf}
  \caption{Sensor falsification attacks can be characterized in two axes:intention and time sensitivity. 
  The attack IDs are explained in Table~\ref{tab:attacks}.}
  \label{fig:taxonomy}
\end{figure}