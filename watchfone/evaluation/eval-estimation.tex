We evaluated \watchfone's estimation accuracy in the absence of attacks. We compare the estimated and the ground truth values for all \numtrips{} trips. 
Fig.~\ref{fig:estimation-accuracy} plots the CDF of the errors. \changes{The results presented in this section corroborate the estimation accuracy reported by prior work \cite{skog_indirect_2014, liu_bigroad:_2017}. We go beyond prior work in our gear estimation evaluation and dig deeper into the common cause of estimation failure for many sensors. Moreover we evaluate the estimation algorithms under normal phone usage scenarios.}

\subsubsection{Speed} 
\watchfone{} can very accurately estimate the speed using GPS sensors 
--- 50th percentile has $<$ \estSpeedMedian{}kmph error, and 95th percentile 
has $<$ \estSpeedHigh{}kmph error. The low-frequency speed 
($<$ 1Hz) is accurately estimated using the GPS-inferred speed and the high-frequency speed ($>$ 1Hz) is estimated using 
the aligned accelerometer readings.

\subsubsection{Gear}
\watchfone{} also accurately estimates the gear position. Over $\approx$80\% of the time, it can exactly estimate the current gear position, and at 95\% percentile, it is wrong by 
1 gear position. 
We observed that the errors are related to where the 
driver travels. \watchfone{} can more accurately estimate the gear 
position when the trip is predominantly in the highway versus 
surface local roads (see Figs.~\ref{fig:local gear error}~and~\ref{fig:highway gear error}). 
We divided the data from the highway 
and the local road by map-matching the GPS points to the 
OpenStreetMap database. This effect occurs because the driver 
tends to stay in the same gear in the highway whereas there is 
much more fluctuation caused by start and stop behavior on the 
surface streets. This is further corroborated by the fact that
gear estimation error and vehicle speed have a negative pairwise correlation of \estGearSpeedCorrelation{}, meaning as the vehicle goes faster, there is less gear estimation error.


\subsubsection{Odometer, Fuel-Level}
\watchfone{} can also accurately estimate 
odometer (50th\% $<$ \estOdometerMedian{}km, 95\% $<$ \estOdometerHigh{}km) and fuel level 
(50th\% $<$ \estFuelMedian{}L, 95\% $<$ \estFuelHigh{}L). Due to the accumulative 
nature of fuel estimation, we noticed an increasing error as 
the trip continues for a longer period of time. The fuel-level 
estimate and drifting error are shown in Fig.~\ref{fig:fuel-error}. The accumulating error only affects the estimates within a single trip. In between trips, \watchfone{} estimates the \textit{change} in odometer and fuel level, and is therefore able to detect injection attacks which happen during the trip.


\subsubsection{Steering Wheel Angle} 
\watchfone{} can estimate steering wheel angle with $<$ \estSteeringMedian{}$^{\circ}$
error in 50\% of all trips, and $<$ \estSteeringHigh{}$^{\circ}$ in 95\% of all trips. 
We found a strong negative correlation between steering wheel angle estimation and vehicle speed (coef=\estSteeringSpeedCorrelation{}) and gear position (coef=\estSteeringGearCorrelation{}).
We found large errors in steering wheel angle when the car moves slowly. This occurs because 
as the car drives faster, there is a stronger relationship between the induced yaw rate in
the vehicle 
body and the steering wheel angle. 

\subsubsection{Engine RPM} 
\watchfone{} has the most challenge in estimating the engine RPM.
The 50\% error is \estRpmMedian{} RPM and 95\% error is \estRpmHigh{} RPM. We use the
estimated vehicle speed and vehicle gear position to calculate the likely RPM value 
(equation shown in Table~\ref{tab:estimation-summary-table}). Even if we use the ground 
truth vehicle speed and gear position, there is a large variability in the calculated
RPM, as shown in Fig.~\ref{fig:ideal-rpm-error}. 

This large variation in the engine RPM estimation is caused by tire slip. The relationship between
the engine RPM and tire speed is mediated by the tire slip ratio \cite{cho_cps_2015}. The tire 
slip is affected by such factors as the road conditions, wear of the tire, and the friction coefficient 
between the tire and the road. To uncover this relationship, we separated the regions with high 
acceleration of the vehicle, where there is likely to be high tire slip as the vehicle moves 
faster, from regions with low acceleration. This difference in the estimation error between the 
two scenarios is shown in Fig.~\ref{fig:tire-slip-pdfs}. We found +524 more RPM error (a 1790\% 
increase) for sudden acceleration versus normal acceleration conditions. Similarly, we found 372\% 
increased error in up-hill versus flat roads and 178\% increased error in high-precipitation areas.
(Not shown in figure)

\begin{figure*}[!htb]
  \centering
   \begin{subfigure}[b]{0.15\textwidth}
      \includegraphics[width=\textwidth]{watchfone/figures/passenger-experiments/passenger_speed-0.pdf}
      \caption{50th - Speed}
   \end{subfigure}
   %
   \begin{subfigure}[b]{0.15\textwidth}
      \includegraphics[width=\textwidth]{watchfone/figures/passenger-experiments/passenger_steering-0.pdf}
      \caption{50th - Steering}
   \end{subfigure}
   %
   \begin{subfigure}[b]{0.15\textwidth}
      \includegraphics[width=\textwidth]{watchfone/figures/passenger-experiments/passenger_rpm-0.pdf}
      \caption{50th - RPM}
   \end{subfigure}
   %
   \begin{subfigure}[b]{0.15\textwidth}
   \RaggedRight
      \includegraphics[width=\textwidth]{watchfone/figures/passenger-experiments/passenger_speed-1.pdf}
      \caption{95th - Speed}
   \end{subfigure}
   %
   \begin{subfigure}[b]{0.15\textwidth}
      \includegraphics[width=\textwidth]{watchfone/figures/passenger-experiments/passenger_steering-1.pdf}
      \caption{95th - Steering}
   \end{subfigure}
   %
   \begin{subfigure}[b]{0.15\textwidth}
      \includegraphics[width=\textwidth]{watchfone/figures/passenger-experiments/passenger_rpm-1.pdf}
      \caption{95th - RPM}
   \end{subfigure}

   \caption{Estimation error using the passenger's phone. The black horizontal line shows 
   the 50th and 95th percentile errors for each of these sensors when the phone is more 
   carefully mounted, as done in our previous experiments.}
   \label{fig:passenger-estimation-accuracy}
\end{figure*}


\subsubsection{GPS blockage noise}
\watchfone\ relies on GPS to estimate the vehicle speed. 
We evaluated the effect of GPS error/noise by adding artificial noise to the dataset before estimating the vehicle speed.
The results are plotted in Fig.~\ref{watchfone rebuttal gps in paper}. 

\begin{figure}
    \centering
    \includegraphics[width=0.4\linewidth]{watchfone/revision/alpha_search.png}
    \caption{Vehicle speed estimation error for increasing high-frequency GPS noise.
    For higher GPS noise, \watchfone\ relies more and more on the accelerometer component 
    of the complementary filter.}
    \label{watchfone rebuttal gps in paper}
\end{figure}

We added normally-distributed noise to the GPS samples 
(sampled up to once every 100ms). We injected random noise for each 100ms time sample of the GPS signal. Realistic 
GPS noise is likely to be less jittery and noisy, but our analysis shows the effect
of a more extreme noise distribution. We found that as we increase the high-frequency 
GPS noise, the average error also increases. The trend is that for every additional meter 
of high-frequency GPS noise, we see an additional error of 1 km/h speed estimation on 
average. Furthermore, we found that with an increased GPS noise, \watchfone\ also relied 
more on the vehicle-aligned accelerometer. This is shown as the $\alpha$ curve in our 
analysis results. $\alpha$ refers to the trade-off between 
GPS-speed and accelerometer-speed, used in the complementary 
filter in \watchfone. To extend this work, we can use the GPS
confidence returned by GPS chips to conditionally turn
on or off \watchfone's estimation capabilities.


\subsubsection{Additional Redundancy from Passengers' Phones}
\label{eval/passenger-phone}

\watchfone\ can be easily extended to make use of additional sources of 
redundancy from other devices. We demonstrate this possibility by running 
\watchfone{} on passengers' smartphones while they engage in various common 
activities on their phone. We recruited 7 different passengers to repeat a 6.7
mile-long circuit four times. Each trip around the circuit, the passenger 
engaged in one of the following four activities --- watching a movie, browsing 
a website, typing out a message, or making a phone call. For each condition we 
collected $\approx$2 hours and 50 miles of data. In total we collected 7.9 hours
and 200 miles of data.


The 50th and 95th percentile estimation errors are shown in 
Fig.~\ref{fig:passenger-estimation-accuracy}. We skipped odometer and fuel 
since those only depend on the GPS, which is unaffected by who is using the 
phone. We also skipped gear estimation results since they are the same as the
50th and 95th percentile estimation accuracy presented earlier in this section.

Across all sensors, the estimation accuracy is only minimally affected by the
passenger using the phone. There are two main factors which affect the 
estimation accuracy while the passenger uses the phone. 

The first factor is how much the user interacts with the phone. We found that 
as the user types on the phone or makes a phone call, the steering wheel 
estimation accuracy tends to be worse. Especially when the passenger types on 
the phone, the 95th percentile steering wheel estimation error is worse compared 
to other activities. The steering wheel estimation heavily makes use of the IMU
sensors to both align to the world-frame and convert the gyroscope readings to 
steering wheel estimates. Increased noise in the IMU readings results in worse 
estimations. Speed is also affected by typing (95th percentile texting is worse
than browsing) but this is overshadowed by a second factor.

The second factor is the phone orientation. When the passenger watches a movie 
(in landscape orientation) or places a phone call, the phone is no longer oriented
in the direction of the car's travel. For both browsing and texting, the phone
is in portrait orientation and faces the direction of the car's travel. Since 
speed estimation uses the accelerometer in the direction of the car's travel, 
this is more sensitive to the phone orientation. 

The second factor doesn't affect steering wheel estimations as badly because 
the phone only needs to the re-oriented to the world-plane to get the angular 
movement. We simply do this using the gravity vector. In contrast,
the speed estimation requires that we get the oriented acceleration along the
axis of the car's movement, which is a more constrained estimation of the 
phone's orientation.

\watchfone{} can similarly be extended using other IoT devices such as the 
driver's smartwatch.

% !TeX root = ./main.tex

\begin{figure}[h]
    \centering
    \begin{tabular}[c]{ccc}
        \multirow{2}{*}[14pt]{
            \begin{subfigure}{0.3\textwidth}
                \vspace{-0.75in}
                \includegraphics[width=\linewidth]{watchfone/figures/06-a-gear_nn-T21192-pilot.pdf}
                \caption{Local road gear error}
                \label{fig:local gear error}
            \end{subfigure}
        } & \multirow{2}{*}[14pt]{
            \begin{subfigure}{0.3\textwidth}
                \vspace{-0.75in}
                \includegraphics[width=\linewidth]{watchfone/figures/06-b-gear_nn-T21050-pilot.pdf}
                \caption{Highway road gear error}
                \label{fig:highway gear error}
            \end{subfigure}
        } & \begin{subfigure}[c]{0.3\textwidth}
                \includegraphics[width=\linewidth]{watchfone/figures/04-all-good-gps/ideal_rpm-workspace-cdf.pdf}
                \caption{Engine RPM estimation error using the ground truth 
                vehicle speed and gear position.}
                \label{fig:ideal-rpm-error}
            \end{subfigure} \\ 
        & & \begin{subfigure}[c]{0.3\textwidth}
                \includegraphics[width=\linewidth]{watchfone/figures/18-tire-slip.pdf}
                \caption{Distribution of RPM errors. Figure best seen in color.}
                \label{fig:tire-slip-pdfs}
            \end{subfigure} \\
        \multicolumn{3}{c}{
            \begin{subfigure}{\linewidth}
                \centering
                \vspace{0.4in}
                \includegraphics[width=0.65\linewidth]{watchfone/figures/05-fuel-T48.pdf}
                \caption{Example comparison of estimated fuel level with the ground truth.}
                \label{fig:fuel-error}
            \end{subfigure} 
        }
    \end{tabular}    
    \caption{\ref{fig:local gear error} and \ref{fig:highway gear error} show the gear estimation error on the highway compared to local roads. The pie chart shows the distribution of the trip on the highway and local roads. The middle line graph depicts the fluctuation of the gear position during the trip. The bottom line graph is the estimation error. \watchfone's estimation error is due to the highly fluctuating and complex behavior of the transmission control unit on the local roads.}
    \label{fig:ceo}
  \end{figure}



\subsubsection{Key Findings}
\changes{We corroborated the errors reported in prior work in estimating vehicle sensors \cite{liu_bigroad:_2017, cho_cps_2015}. We also made three key 
findings, summarized below.}

\begin{itemize}
    \item \changes{Phone-based estimation algorithms work reliably even while the phone is being used for common activities such as texting or making a phone call.}
    
    \item \changes{Gear position can be accurately estimated using a neural network that accepts the recent change in vehicle speed.} 
    
    \item \changes{By conditioning for various factors, we discovered that engine RPM estimation is plagued by tire slip. This is especially predominant in steep road segments or regions with high-acceleration applied to the car.}
\end{itemize}

% Our key findings in estimating 6 vehicular sensors using 3 phone sensors can be summarized as follows. The first three key findings reflect and corroborate similar findings in related work. The fourth key finding is an extension based on our experiments with passenger phones.
% \begin{itemize}
%     \item The phone can be used to accurately estimate the speed, steering wheel, gear position, consumed fuel level, and change in odometer.
    
%     \item It is difficult to estimate engine RPM. This value cannot be directly estimated from the other sensors due to tire slip caused by external factors.
    
    
%     \item \watchfone{} can be extended to other sensing devices such as the passenger's phone. The estimation accuracy only minimally suffers while the passenger engages in common activities on their phone. 
% \end{itemize}